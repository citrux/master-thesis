\begin{titlepage}
\begin{center}
    Министерство образования и науки РФ\\
    Федеральное государственное бюджетное образовательное учреждение высшего образования\\
    <<ВОЛГОГРАДСКИЙ ГОСУДАРСТВЕННЫЙ ТЕХНИЧЕСКИЙ УНИВЕРСИТЕТ>>\\
    \vspace{3cm}
    Абдрахманов Владимир Леватович\\
    \vspace{1cm}
    <<Влияние постоянного электрического поля на эффект двухволнового смешивания в графеновой сверхрешётке>>\\
    \vspace{3cm}
    Диссертация на соискание учёной степени магистра физики по магистерскому направлению 03.04.02 <<Физика>> (профиль магистерской подготовки <<Физика радиоэлектронных технологий>>)\\
    \vspace{3cm}
    \begin{flushright}
        \begin{minipage}{0.6\textwidth}
            \begin{center}
                Научный руководитель\\
                доктор физ.-мат. наук Завьялов~Д.~В.
            \end{center}
        \end{minipage}
    \end{flushright}
    \vfill
    Волгоград, 2016
\end{center}
\newpage
\thispagestyle{empty}


    \begin{center}
       Министерство образования и науки РФ\\
    Федеральное государственное бюджетное образовательное учреждение высшего образования\\
    <<ВОЛГОГРАДСКИЙ ГОСУДАРСТВЕННЫЙ ТЕХНИЧЕСКИЙ УНИВЕРСИТЕТ>>\\
        \vspace{.5cm}
        Кафедра <<Физика>>
    \end{center}
    \vspace{1cm}
    \begin{flushright}
        \newsavebox\tmp
        \sbox\tmp{Заведующий кафедрой <<Физика>>}%
        \begin{minipage}{\wd\tmp}
        \begin{center}
        УТВЕРЖДАЮ\\
        \vspace{.3cm}
        \box\tmp
        \vspace{.3cm}
        \hrulefill\hspace{1em}Завьялов~Д.~В.\\
        \vspace{.3cm}
        <<\underline{\hspace{.5cm}}>>~\underline{\hspace{3cm}}~\the\year~г.
        \end{center}
        \end{minipage}
    \end{flushright}
    \vspace{2cm}
    \begin{center}
        \Large \textbf{ПОЯСНИТЕЛЬНАЯ ЗАПИСКА} \\
        \large к магистерской диссертации на тему
    \end{center}
    \vspace{1cm}
    \begin{center}
        <<Влияние постоянного электрического поля на эффект двухволнового смешивания в графеновой сверхрешётке>>
    \end{center}
    \vspace{2cm}
    \begin{flushleft}
        Автор\hspace{2.5cm}\underline{\hspace{5cm}}\hspace{1cm}Абдрахманов~В.~Л.\\
        \vspace{.5cm}
        Обозначение\hspace{1cm}\underline{\hspace{5cm}}\\
        \vspace{.5cm}
        Группа\hspace{2.2cm}Ф-2н\\
        \vspace{.5cm}
        Направление\hspace{1cm}03.04.02 Физика (Физика радиоэлектронных технологий)\\
        \vspace{.5cm}
        Руководитель работы\hfill\underline{\hspace{5cm}}\hfillЗавьялов~Д.~В.
    \end{flushleft}

    \vspace{\fill}

    \begin{center}
        Волгоград, \the\year
    \end{center}
    \newpage
    \thispagestyle{empty}
    \begin{center}
       Министерство образования и науки РФ\\
    Федеральное государственное бюджетное образовательное учреждение высшего образования\\
    <<ВОЛГОГРАДСКИЙ ГОСУДАРСТВЕННЫЙ ТЕХНИЧЕСКИЙ УНИВЕРСИТЕТ>>\\
        \vspace{.5cm}
        Кафедра <<Физика>>
    \end{center}
    \vspace{1cm}
    \begin{flushright}
        \sbox\tmp{Заведующий кафедрой <<Физика>>}%
        \begin{minipage}{\wd\tmp}
        \begin{center}
        УТВЕРЖДАЮ\\
        \vspace{.3cm}
        \box\tmp
        \vspace{.3cm}
        \hrulefill\hspace{1em}Завьялов~Д.~В.\\
        \vspace{.3cm}
        <<\underline{\hspace{.5cm}}>>~\underline{\hspace{3cm}}~\the\year~г.
        \end{center}
        \end{minipage}
    \end{flushright}
    \vspace{1cm}
    \begin{center}
       Задание на магистерскую диссертацию
    \end{center}
    \vspace{.5cm}
    \begin{flushleft}
        Студент Абдрахманов Владимир Леватович\\
        \vspace{.5cm}
        Код кафедры 10.38\hfillГруппа Ф-2н\\
        \vspace{.5cm}
        Тема <<Влияние постоянного электрического поля на эффект двухволнового смешивания в графеновой сверхрешётке>>\\
        \vspace{.3cm}
        Утверждена приказом по ВолгГТУ от <<\underline{\hspace{.5cm}}>> \underline{\hspace{4cm}} №\underline{\hspace{2cm}}\\
        \vspace{.3cm}
        Срок представления готовой работы \hrulefill\\
        \vspace{.3cm}
        Исходные данные для выполнения работы\\
        Спектр графеновой сверхрешётки, уравнение Больцмана, сила Лоренца.\\
        \vspace{.3cm}
        Содержание основной части пояснительной записки\\
        Введение, Электронный транспорт в сверхрешётках (литературный обзор), Влияние постоянного поля на взаимное выпрямление в графеновой сверхрешётке, Заключение.
        \vspace{.3cm}
        \begin{center}
            Перечень графического материала\\
            \hrulefill\\
        \end{center}
        \vspace{1cm}
        Руководитель работы\hfill\underline{\hspace{5cm}}\hfillЗавьялов~Д.~В.
    \end{flushleft}
\end{titlepage}