\ssection{Введение}
Исследование совместного воздействия высокочастотных и постоянных электромагнитных полей на низкоразмерные структуры \cite{olbrich, hyart, neg-conductivity, mutual-rectification, liu, attaccalite, silveirinha}  представляет интерес в связи с возможностью расширить представления о свойствах этих материалов при помощи сравнительно простых в постановке экспериментов. 

Для построения детекторов высокочастотного электромагнитного излучения могут быть использованы эффекты когерентного смешивания электромагнитных волн \cite{mensah, hendry, kumar,two-axis-sl}, состоящие в генерации постоянной составляющей плотности тока в под действием высокочастотного электрического поля и его кратной (чётной) гармоники. Одной из первых работ, посвященных подобным эффектам, является \cite{mensah}, где рассмотрена одомерная полупроводниковая сверхрешетка (СР), перпендикулярно оси которой распространяются две электромагнитные волны, поляризованные вдоль оси СР.

Одномерные полупроводниковые СР, как правило, обладают аддитивным спектром. В противоположность им, ГСР демонстрируют неаддитивность спектра, обусловленную ``релятивистским'' спектром графена: движение носителей зависит как от продольного воздействия вдоль направления движения, так и от поперечного. Этот факт позволяет наблюдать в таких СР взаимное выпрямление как вдоль оси СР, так и поперёк. Более того, в ГСР возможно взаимное выпрямление между волнами с ортогональными плоскостями поляризации \cite{gsl-sio2-sic}.

Влияние постоянного поля на двухволновое выпрямление в структуре с неаддитивным спектром на примере двухмерной полупроводниковой СР рассматривалось в \cite{two-axis-sl}. Однако, в этом случае и постоянное электрическое поле, и высокочастотные поля были ориентированы вдоль осей СР, что несколько осложняет выявление физических причин, приводящих к наблюдаемой резонансной зависимости тока от постоянного поля.

Всё вышенаписанное обуславливает актуальность работы.

Цель работы заключается в исследовании влияния постоянного электрического поля на явления взаимного выпрямления в одномерной ГСР под воздействием двух линейно поляризованных волн с кратными частотами и общей плоскостью поляризации.

Для достижения цели поставлены следующие задачи:
\begin{itemize}
    \item получить зависимость тока продольного двухволнового выпрямления от поперечного постоянного электрического поля;
    \item получить зависимость тока поперечного двухволнового выпрямления от продольного постоянного электрического поля;
    \item изучить характер изменения вида зависимостей при изменении частоты высокочастотного поля.
\end{itemize}

Научная новизна состоит в том, что автором впервые получена зависимость тока при двухволновом выпрямлении от напряженности поперечного постоянного электрического поля в ГСР на полосчатой подложке.