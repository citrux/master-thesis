\section*{Реферат}
Исследовано влияние постоянного поперечного поля на эффект двухволнового смешивания в одномерной графеновой сверхрешётке (ГСР). Рассмотрены две ориентации полей: 1) плоскость поляризации содержит ось ГСР и 2) плоскость поляризации перпендикулярна оси ГСР. В обоих случаях постояное поле перпендикулярно плоскости поляризации. Плотность тока рассчитана в одноминизонной модели при помощи решения уравнения Больцмана в приближении постоянного времени релаксации.

\vspace{1cm}\noindent\nohyphens{%  
Ключевые слова: взаимное выпрямление, двухволновое смешивание, графеновая сверхрешётка, уравнение Больцмана, приближение постоянного времени релаксации, квазиклассическое приближение, одноминизонное приближение.}

\vspace{3cm}\selectlanguage{english}
We study the direct current density, produced by two-wave mutual rectification, subjected to transverse static uniform electric field in uniaxial graphene superlattice (GSL). We consider two different field orientations: 1) the vector of polarization is directed along an axis of GSL and 2) the vector of polarization is perpendicular axes GSL. In both cases the vector of polarization is perpendicular to constant electric field. Current density is calculated in one-miniband model by solving Boltzmann kinetic equation with constant relaxation time approximation.

\vspace{1cm}\noindent\nohyphens{%  
Keywords: mutual rectification, two-wave mixing, graphene superlattice, Boltzmann equation, constant relaxation time approximation, semiclassical approximation, one-miniband model.}\selectlanguage{russian}