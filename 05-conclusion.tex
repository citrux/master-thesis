\ssection{Заключение}
В работе была исследована зависимость тока взаимного выпрямления от перпендикулярного постоянного поля в графеновой сверхрешётке на полосчатой подложке. Были получены следующие результаты:
\begin{enumerate}
    \item При продольном двухволновом выпрямлении приложенное поперечное постоянное электрическое поле приводит к уменьшению величины тока при малых частотах ВЧ поля (\(\omega\tau < 1\)). При  больших частотах (\(\omega\tau > 1\)) наблюдается увеличение тока в слабом поле, сменяющееся уменьшением в сильных полях. Резонансных явлений, подобных наблюдаемым в двухмерной СР, не обнаружено.
    \item При поперечном двухволновом выпрямлении приложенное продольное постоянное электрическое поле при малых частотах ВЧ поля (\(\omega\tau < 1\)) также приводит к уменьшению величины тока. Однако, при больших частотах (\(\omega\tau > 1\)) наблюдается резкое увеличение тока и смена его направления при приближении штарковской частоты к частоте одной из волн. По всей видимости, это связано с наложением блоховских осцилляций, вызванных постоянным электрическим полем, и колебаний под действием ВЧ поля.
    \item С ростом частоты падающих волн ток взаимного выпрямления убывает пропорционально частоте, по крайней мере в одноминизонном приближении, когда энергия оптических квантов меньше ширины щели между минизонами.
\end{enumerate}